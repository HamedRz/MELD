\documentclass[]{article}
\usepackage{lmodern}
\usepackage{amssymb,amsmath}
\usepackage{ifxetex,ifluatex}
\usepackage{fixltx2e} % provides \textsubscript
\ifnum 0\ifxetex 1\fi\ifluatex 1\fi=0 % if pdftex
  \usepackage[T1]{fontenc}
  \usepackage[utf8]{inputenc}
\else % if luatex or xelatex
  \ifxetex
    \usepackage{mathspec}
  \else
    \usepackage{fontspec}
  \fi
  \defaultfontfeatures{Ligatures=TeX,Scale=MatchLowercase}
\fi
% use upquote if available, for straight quotes in verbatim environments
\IfFileExists{upquote.sty}{\usepackage{upquote}}{}
% use microtype if available
\IfFileExists{microtype.sty}{%
\usepackage{microtype}
\UseMicrotypeSet[protrusion]{basicmath} % disable protrusion for tt fonts
}{}
\usepackage{hyperref}
\hypersetup{unicode=true,
            pdfborder={0 0 0},
            breaklinks=true}
\urlstyle{same}  % don't use monospace font for urls
\IfFileExists{parskip.sty}{%
\usepackage{parskip}
}{% else
\setlength{\parindent}{0pt}
\setlength{\parskip}{6pt plus 2pt minus 1pt}
}
\setlength{\emergencystretch}{3em}  % prevent overfull lines
\providecommand{\tightlist}{%
  \setlength{\itemsep}{0pt}\setlength{\parskip}{0pt}}
\setcounter{secnumdepth}{0}
% Redefines (sub)paragraphs to behave more like sections
\ifx\paragraph\undefined\else
\let\oldparagraph\paragraph
\renewcommand{\paragraph}[1]{\oldparagraph{#1}\mbox{}}
\fi
\ifx\subparagraph\undefined\else
\let\oldsubparagraph\subparagraph
\renewcommand{\subparagraph}[1]{\oldsubparagraph{#1}\mbox{}}
\fi

\date{}

\begin{document}

\section{MELD}\label{meld}

This repository contains a python package that implements \emph{MELD}, a
fast moment estimation method for generalized Dirichlet latent variable
model. For the details of the method and the model, please see
http://arxiv.org/abs/1603.05324.

\section{Introduction}\label{introduction}

MELD stands for \emph{M}oment \emph{E}stimation for generalized
\emph{L}atent \emph{D}irichlet variable models.

\subsection{The model}\label{the-model}

The generalized latent Dirichlet variable model in MELD assumes the
$p$ dimensional observation $\boldsymbol{y}_i$
is a mixture of $k$ latent components, with mixture weight denoted
as a Dirichlet distributed latent variable $\boldsymbol{x}_i$.
This model is also known as a \emph{mixed membership model}. In contrast
to previous mixed membership models, the new model allows each
coordinate of $\boldsymbol{y}_i$ to take different variable
types, including categorical, continuous and integer-valued variables.

\subsection{Parameter estimation}\label{parameter-estimation}

The parameter estimation method developed in MELD uses a moment method
known as generalized method of moments (GMM). This method is in contrast
to previous parameter estimation approaches such as MCMC or EM
algorithms that require instantiations of latent variables. The new GMM
approach does not require instantiations of latent variables. By
encoding each coordinate of $\boldsymbol{y}_i$ by a dummy
variable, our new approach calculates the \emph{cross} moment matrices
or tensors among variables, with latent variables effectively
marginalized out. Parameter estimation is conducted using a fast
coordinate descent algorithm.

\section{About the folders}\label{about-the-folders}

\begin{itemize}
\item
  \texttt{data} contains the simulated categorical data with $n =
  \{50,100,200,500,1000\}$ observations. The number of categorical
  variables is $20$. Each categorical variable could take $4$
  different categories.
\item
  \texttt{code} contains the implementation of MELD and a python script
  showing how to use MELD.
\end{itemize}

\end{document}
